\documentclass[12pt]{article}
\usepackage[utf8]{inputenc}
\usepackage{amsmath}
\usepackage{amssymb}
\usepackage{amsthm}

\title{Test Document 2: Mathematical Equations}
\author{Mathematics Test}
\date{\today}

\theoremstyle{definition}
\newtheorem{theorem}{Theorem}
\newtheorem{definition}{Definition}

\begin{document}

\maketitle

\section{Introduction to Mathematics}
This document tests various mathematical typesetting capabilities.

\section{Equations}

\subsection{Inline and Display Equations}
The quadratic formula is $x = \frac{-b \pm \sqrt{b^2 - 4ac}}{2a}$.

\begin{equation}
    \int_{0}^{\infty} e^{-x^2} dx = \frac{\sqrt{\pi}}{2}
\end{equation}

\subsection{Matrix Operations}
\begin{equation}
    A = \begin{bmatrix}
        1 & 2 & 3 \\
        4 & 5 & 6 \\
        7 & 8 & 9
    \end{bmatrix}
\end{equation}

\section{Theorems and Proofs}

\begin{theorem}[Pythagorean Theorem]
In a right triangle, $a^2 + b^2 = c^2$ where $c$ is the hypotenuse.
\end{theorem}

\begin{definition}
A prime number is a natural number greater than 1 that has no positive divisors other than 1 and itself.
\end{definition}

\section{Complex Equations}
\begin{align}
    \nabla \times \vec{E} &= -\frac{\partial \vec{B}}{\partial t} \\
    \nabla \times \vec{B} &= \mu_0 \vec{J} + \mu_0 \epsilon_0 \frac{\partial \vec{E}}{\partial t}
\end{align}

\end{document}